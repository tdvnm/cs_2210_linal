\documentclass[11pt]{article}
\usepackage[utf8]{inputenc}
\usepackage[letterpaper,top=0cm, margin=0.85in]{geometry}

%for math
\usepackage{amsmath, amssymb, amsfonts} %standard
\usepackage{youngtab} % makes squares for math diagrams
%-----------------------------------------------------------           

%\usepackage{sectsty}
%for lists and numbers
\usepackage{enumitem}
%-----------------------------------------------------------

% Doc setting
\usepackage[english]{babel} % Replace `english' with e.g. `spanish' to change the document language
\usepackage{textcmds} %more symbols
\usepackage{fontspec} %more fonts
\usepackage{setspace} %to set spacing bw words and lines
\usepackage{changepage}
\setlength\parindent{0pt}

%footer
\usepackage{fancyhdr}
\usepackage{lastpage}

\fancyhf{} % sets both header and footer to nothing
\renewcommand{\headrulewidth}{0pt} %remove headerline

\fancyfoot[RE,RO]{\thepage}
\fancyfoot[LE,LO]{\emph{Shubhro Gupta — CS-2210}}
\pagestyle{fancy}
%-----------------------------------------------------------

%for pictures and graphs
\usepackage{graphicx} %add image
\usepackage{adjustbox}

\usepackage{pgfplots} %for graphing plotting
\pgfplotsset{compat=1.18, width=10cm}
%-----------------------------------------------------------

%for code
\usepackage{verbatim}
\usepackage{listings}
\usepackage{fancyvrb} %for coding blocks
%\usepackage{algorithm}
%\usepackage{algpseudocode} %for pseudocode
%\usepackage{algorithm, algpseudocode}
\usepackage[linesnumbered]{algorithm2e}
\usepackage{algorithm2e}

\setmonofont[Scale=MatchLowercase]{[SFMono-Regular.ttf]}
%\usepackage{lstfiracode} %firacode
\usepackage[framemethod=tikz]{mdframed} %adding background to lstlisting
\usepackage[ruled,vlined,boxed]{algorithm2e} %for pseudocode lines



%for colors and links
\usepackage[colorlinks = true,
            linkcolor = blue,
            urlcolor  = blue,
            citecolor = blue,
            anchorcolor = blue]{hyperref}
\usepackage[many]{tcolorbox}  % for colored boxes
\usepackage{color} % to get colors
\usepackage{xcolor} %more colors options and flexibility
\usepackage{transparent}



%-----------------------------------------------------------------------------


%CUSOMIZATIONS

% my colors
%dracula
\definecolor{background}{rgb}{0.16,0.16,0.21}
\definecolor{codegreen}{rgb}{0.24,0.68,0.65}
\definecolor{codepurple}{rgb}{0.51,0.31,0.87}
\definecolor{codered}{rgb}{0.81,0.13,0.18}
\definecolor{codebluegray}{rgb}{0.02,0.31,0.68}

\definecolor{comment}{rgb}{0.67,0.74,0.79}
\definecolor{textcolor}{rgb}{0.22,0.22,0.22}

%style for coding
\lstdefinestyle{python}{
    language=python,
    backgroundcolor=\color{white},   
    commentstyle={\color{comment}},
    keywordstyle={\color{codepurple}},
    stringstyle={\color{codegreen}},
    basicstyle={\ttfamily\color{textcolor}},
    keywordstyle = [2]{\color{codered}},
    keywordstyle = [3]{\color{codebluegray}},
    keywordstyle = [4]{\color{teal}},
    otherkeywords = {<, >, +, -, =, *, \[, \], &&, ||, format, 1, 2, 3, 4, 5, 6, 7, 8, 9, 0, ;},
    morekeywords = [4]{+, -, *, /, =, <, >, format},
    morekeywords = [3]{\[, \],  1, 2, 3, 4, 5, 6, 7, 8, 9, 0, ;},
    %
    breakatwhitespace=false, 
    frame=shadowbox,
    rulecolor=\color{textcolor},
    breaklines=true,                 
    captionpos=b,                    
    keepspaces=true,                 
    numbers=left,
    numbersep=15pt, %distance between code and numbers
    numberstyle=\scriptsize\ttfamily\color{comment},
    showspaces=false,                
    showstringspaces=false,
    showtabs=false,
    xleftmargin=4.3em, %margin bw left page and frame
    framexleftmargin=3.8em, %margin bw text and frame
    %xleftmargin=3.4em,
    framexrightmargin=-0.5em,
    tabsize=2,
    aboveskip=1.5em,
    belowskip=0.5em,
    framextopmargin=9pt,
    framexbottommargin=9pt,
    frameshape={RYR}{Y}{Y}{RYR}
}

\lstdefinestyle{c}{
    language=c,
    backgroundcolor=\color{white},   
    commentstyle={\color{comment}},
    keywordstyle={\color{codepurple}},
    stringstyle={\color{codegreen}},
    basicstyle={\ttfamily\color{textcolor}},
    keywordstyle = [2]{\color{codered}},
    keywordstyle = [3]{\color{codebluegray}},
    keywordstyle = [4]{\color{teal}},
    otherkeywords = {<, >, +, -, =, *, \[, \], &&, ||, stdio.h, stdlib.h, 1, 2, 3, 4, 5, 6,7, 8, 9, 0, ;},
    morekeywords = [4]{+, -, *, /, =, <, >, stdio.h, stdlib.h},
    morekeywords = [3]{\[, \],  1, 2, 3, 4, 5, 6, 7, 8, 9, 0, ;},
    morekeywords = [2]{&&, ||},
    %
    breakatwhitespace=false, 
    frame=shadowbox,
    rulecolor=\color{textcolor},
    breaklines=true,                 
    captionpos=b,                    
    keepspaces=true,                 
    numbers=left,                    
    numbersep=15pt, %distance between code and numbers
    numberstyle=\scriptsize\ttfamily\color{comment},
    showspaces=false,                
    showstringspaces=false,
    showtabs=false,
    xleftmargin=4.3em, %margin bw left page and frame
    framexleftmargin=3.8em, %margin bw text and frame
    %xleftmargin=3.4em,
    framexrightmargin=-0.5em,
    tabsize=2,
    aboveskip=1.5em,
    belowskip=0.5em,
    framextopmargin=9pt,
    framexbottommargin=9pt,
    frameshape={RYR}{Y}{Y}{RYR}
}

%-----------------------------------------------------------------------------
%custom commands

%code
\newcommand{\problem}[1]{\begin{adjustwidth}{0.1px}\noindent \framebox[1.2\width]{\large Problem #1}\end{adjustwidth} \bigskip\\}
\newcommand{\codecap}[2]{{\vspace{4px}{\emph{#1}}} \hfill \href{#2}{Link to the code\ }\vspace{25px}}
\newcommand{\code}[1]{{\texttt{#1}}}

\SetKwInput{KwInput}{Input}
\SetKwInput{KwOutput}{Output}

%math
\newcommand{\bigo}[1]{$O(#1)$ }
\newcommand{\thetan}[1]{$\theta(#1)$}
\newcommand{\vector}[1]{$\overrightarrow{#1}$}

\newcommand{\vecset}[2]{\{ {#1}_1, {#1}_2, {#1}_3,  \dots,  {#1}_{#2}\}}

%display
\newcommand{\link}[3][blue]{\href{#2}{\color{#1}{#3}}}%
\newcommand{\inlink}[1]{\underline{\emph{\link[black]{#1}{#1}}}}


%header
\newcommand{\lesgo}[5]{
\begin{large}
\emph{#1}\smallskip \\
\textbf{Shubhro Gupta} \hfill Week #2\smallskip \\
Professor #3 \hfill Due #4\\
\end{large} \medskip \\
{\emph{Collaborators: #5}}\\
\hrule
\vspace{50px}
\\
}

\newcommand\dunderline[3][-1pt]{{%
  \sbox0{#3}%
  \ooalign{\copy0\cr\rule[\dimexpr#1-#2\relax]{\wd0}{#2}}}}

%new section
\newcommand{\asec}[1]{{\vspace{20px}\large\dunderline[-3px]{1px}{\textbf{#1}}} \\}




%-----------------------------------------------------------------------------
%title
\usepackage{algpseudocode}
\begin{document}

\lesgo{CS-2210 Linear Algebra}{3}{Pritam Ghosh}{October 10, 2022}{none}

\problem{1}
Check whether the following sets are sub-spaces of their respective vector spaces. Make sure to check all 3 axioms and mention which of them fail.
\begin{enumerate}[label=(\Alph*)]
    \item $V = \mathbb{R}^3, \ F = \mathbb{R}, \ W=\left\{\left(x_1, x_2, x_3\right) \mid x_1+2 x_2<0\right\}$.
    \item $V = \mathbb{R}^3, \ F = \mathbb{R}, \ W=\left\{\left(x_1, x_2, x_3\right) \mid x_3=x_1+ x_2\right\}$.
    \item $V = \mathbb{R}^3, \ F = \mathbb{R}, \ W=\left\{\left(x_1, x_2\right) \mid x_1+ x_2=1\right\}$.
    
    \item $V = \mathbb{R}^3, \ F = \mathbb{R}, \ W=\left\{\left(x_1, x_2, x_3\right) \mid 
\begin{bmatrix}
x_1 \\ x_2 \\ x_3
\end{bmatrix} = 
s \begin{bmatrix}
2 \\ 1 \\ 1
\end{bmatrix}
+ t
\begin{bmatrix}
1 \\ 2 \\ 1
\end{bmatrix}, \text{ for }s,t \in \mathbb{R}
    \right\}$.
    
    \item $V = \mathbb{R}^3, \ F = \mathbb{R}, \ W=\left\{\left(x_1, x_2, x_3\right) \mid 
    \begin{bmatrix}
x_1 \\ x_2 \\ x_3
\end{bmatrix} = 
\begin{bmatrix}
0 \\ 1 \\ 2
\end{bmatrix} + 
s \begin{bmatrix}
2 \\ 1 \\ 1
\end{bmatrix}
+ t
\begin{bmatrix}
1 \\ 2 \\ 1
\end{bmatrix}, \text{ for }s,t \in \mathbb{R}
    \right\}$.
    
    \item $V = \mathbb{R}^3, \ F = \mathbb{R}, \ W=\left\{\left(x_1, x_2, x_3\right) \mid 
    \begin{bmatrix}
x_1 \\ x_2 \\ x_3
\end{bmatrix} = 
\begin{bmatrix}
\llap{-}1 \\ 1 \\ 0
\end{bmatrix} + 
s \begin{bmatrix}
2 \\ 1 \\ 1
\end{bmatrix}
+ t
\begin{bmatrix}
1 \\ 2 \\ 1
\end{bmatrix}, \text{ for }s,t \in \mathbb{R}
    \right\}$.
    
    \item $V = \mathbb{R}^3n, \ F = \mathbb{R}, \ A \text{ is }m \times n \text{ matrix}, \ W=\{ x \in \mathbb{R}^n \mid A\vec{x} = 3\vec{x} \}$.
    \item $V = \mathbb{R}^3n, \ F = \mathbb{R}, \ A \text{ is }m \times n \text{ matrix}, \ W=\{ x \in \mathbb{R}^n \mid A\vec{x} = \vec{0} \}$.
\end{enumerate}
\bigskip\\
\textbf{Solution}\\




\newpage
\problem{2}
\begin{flalign*}
&\begin{aligned}
\text{Let } \zeta^0[0, 1] & = \text{Continuous functions } f:[0, 1] \rightarrow R\\
\zeta^1[0, 1] & = \text{Differentiable functions } f:[0, 1] \rightarrow R\\
\zeta^2[0, 1] & = \{ f:[0, 1] \rightarrow R \mid f"(x) \text{ exists for all } x \in [0, 1]\}.
\\
P_k & = \text{All polynomials with degree} \leq k \text{ and have real co-efficients}.
\end{aligned}&&
\end{flalign*}
\begin{enumerate}[label=(\Alph*)]
    \item Show that $\zeta^0[0, 1], \ \zeta^1[0, 1], \  \zeta^2[0, 1], $ and $P_k$ are all vector spaces.
    \item Show that $\zeta^1[0, 1]$ is a \textbf{proper} subspace of $\zeta^1[0, 1]$.
    \item If $\zeta(\mathbb{R})$ denotes a continous function $f: \mathbb{R} \rightarrow \mathbb{R}$, show that $P_k$ is a sub-space $\zeta(\mathbb{R})$, $\zeta^1(\mathbb{R})$.
    \item Is $W =  \{ f(x) \in P_k \mid \int_{0}^{1} f(t) dt = 0\}$ a sub-space of $P_k$?
    
    \item Is $W =  \{ f(x) \in \zeta^1(\mathbb{R}) \mid f't + 2f(t) = 0 \forall t\}$ a sub-space of $\zeta(\mathbb{R})$?
    
    \item Is $W =  \{ f(x) \in \zeta[0, 1] \mid f(1) = 2 \}$ a sub-space of $\zeta[0, 1]$?
    
\end{enumerate}
\bigskip\\
\textbf{Solution}






\newpage
\problem{3}
Let $V$ be any vector space over $\mathbb{R}$.
\begin{enumerate}[label=(\Alph*)]
    \item If $U, \ W \subset V$ are sub-spaces of $V$, then show that $U \cap W$ is also a sub-space of $V$.
    \item If $U, \ W \subset V$ are sub-spaces of $V$, then give an example to show that $U \cup W$ may not be a sub-space of $V$.
    \item Let $U, \ W$ be any sub-space of $V$.\\
    Let $U+W = \{ \vec{u} + \vec{w} \in V \mid \vec{u} \in U, \ \vec{w} \in W \}$.\\
    Show that $U+W$ is a sub-space of $W$.
    \item Give two examples where $U, \ W$ are two proper sub-spaces of some vector space $V$ such that 
    \begin{enumerate}[label=(\alph*)]
        \item $U+W$ is a \textbf{proper }sub-space of $V$.
        \item $U+W=V$
    \end{enumerate}
    \item Show that $(U+W)^\perp = U^\perp \cap W^\perp$, where $U, \ W$ are sub-spaces of $V$, where $V$ is an inner product space.
\end{enumerate}
\bigskip\\
\textbf{Solution}








\newpage
\problem{4}
\begin{enumerate}[label=(\Alph*)]
    \item Show that if $V = $ Z-axis in $\mathbb{R}^3$ and $W = $ XY-plane in $\mathbb{R}^3$, then $W = V^{\perp}$ and $V \cap W = \{ \vec{0}\}$.
    \item Suppose $V, \ W$ are sub-spaces of $\mathbb{R}^n$, such that $V^{\perp} = W$. Then show that $V \cap W = \{ \vec{0}\}$.
    \item Suppose $V, \ W$ are sub-spaces of $\mathbb{R}^n$ with $V \subset W$. Then show that $W^\perp \subset V^\perp$.
    \item Draw a schematic picture in $\mathbb{R}^3$, with $V = $ X-axis, $W = $ XY-plane to demonstrate problem(C).
\end{enumerate}
\bigskip\\
\textbf{Solution}











\newpage
\problem{5}
Let $P_k$ denote the vector space of all polynomials of degree $\leq k$, having real co-efficients.
\begin{enumerate}[label=(\Alph*)]
    \item Show that $\{ 1, x, x^2, x^3, \dots, x^k\}$ is a basis of $P_k$.
    \item Show that $\langle f, g\rangle = \int_{a}^{b} f(t) \cdot g(t) \ dt$ defines an inner-product on $P_k$ for some fixed $a, b \in \mathbb{R}$.
\end{enumerate}
\bigskip\\
\textbf{Solution}










\newpage
\problem{6}
\begin{enumerate}[label=(\Alph*)]
    \item Are the vectors $\left\{ f(x) = 1, \ g(x) = cos(x), \ h(x) = sin(x)\right\}$ linearly independent in $\zeta[0, \ 2\pi]$?
    \item Are the vectors $f(x), \ g(x)$ orthogonal? What about $g(x), \ h(x)$?
\end{enumerate}
\bigskip\\
\textbf{Solution}










\newpage
\problem{7}
Consider the function $g(x) = x - 1 \in [0, \ 1]$. What is the orthogonal complement of the sub-space spanned by $g(x)$ in [0, 1]?
\bigskip\\
\textbf{Solution}












\newpage
\problem{8}
Let $V \in \mathbb{R}^n$ be a sub-space. \\
Show that ($v^{\perp})^{\perp} = V$.
\bigskip\\
\textbf{Solution}











\newpage
\problem{9}
Let $W = span\left\{ \begin{bmatrix}
1 \\ 1 \\ 1 \\ 2
\end{bmatrix},  \begin{bmatrix}
1 \\ \llap{-}1 \\ 5 \\ 2
\end{bmatrix}\right\} \in \mathbb{R}^4$.\\
Describe $W^\perp$ and write down a basis for $W^\perp$.
\bigskip\\
\textbf{Solution}













\newpage
\problem{10}
Let $V$ be any vector space over $\mathbb{R}$. Suppose $\{\vec{u}, \vec{v}, \vec{w}\}$ is linearly independent set of vectors. 
\\ Show that $\{\vec{u} + \vec{v}, \vec{v} - \vec{w}, \vec{w}\}$ is also linearly independent. \\
Is $span\{\vec{u}, \vec{v}, \vec{w}\} = span\{\vec{u} + \vec{v}, \vec{v} - \vec{w}, \vec{w}\}$
\bigskip\\
\textbf{Solution}













\newpage
\problem{11}
Check whether the following sets of vector are basis of $V$ mentioned below:
\begin{enumerate}[label=(\Alph*)]
    \item $\left\{ 
    \begin{bmatrix}
    1 \\ 2 \\ 1
    \end{bmatrix},
    \begin{bmatrix}
    1 \\ 2 \\ 1
    \end{bmatrix},
    \begin{bmatrix}
    1 \\ 2 \\ 1
    \end{bmatrix}
    \right\}, \  v = \mathbb{R}^3 \text{ over } \mathbb{R}$.
    
    \item $\left\{ 
    \begin{bmatrix}
    1 \\ 0 \\ 2 \\ 3
    \end{bmatrix},
    \begin{bmatrix}
    0 \\ 1 \\ 1 \\ 1
    \end{bmatrix},
    \begin{bmatrix}
    1 \\ 1 \\ 4 \\ 4
    \end{bmatrix}
    \right\}, \  v = \mathbb{R}^4 \text{ over } \mathbb{R}$.
    
    \item $\left\{ 
    \begin{bmatrix}
    1 \\ 1 \\ 1
    \end{bmatrix},
    \begin{bmatrix}
    2 \\ 3 \\ 3
    \end{bmatrix},
    \begin{bmatrix}
    0 \\ 1 \\ 2
    \end{bmatrix}
    \right\}, \  v = \mathbb{R}^3 \text{ over } \mathbb{R}$.
    
\end{enumerate}
\bigskip\\
\textbf{Solution}















\newpage
\problem{12}
For each of the following matrices $A$, write down:\\
a \underline{basis} of $ker(A)$\\
a \underline{basis} of $Img(A)$\\

\begin{minipage}{0.40\textwidth}
\begin{enumerate}[label=(\Alph*)]
    \item $A = 
    \begin{bmatrix}
    1 & 2 & 3\\
    4 & 5 & 6
    \end{bmatrix}
    $
    \item $A = 
    \begin{bmatrix}
    2 & 1 & 3\\
    4 & 3 & 5\\
    3 & 3 & 5
    \end{bmatrix}
    $
\end{enumerate}
\end{minipage}%
\begin{minipage}{0.40\textwidth}
\begin{enumerate}
    \item [(C)] $A = 
    \begin{bmatrix}
    1 & \llap{-}1 & 1 & 1 & 0\\
    1 & 0 & 2 & 1 & 1\\
    0 & 2 & 2 & 2 & 0\\
    \llap{-}1 & 1 & \llap{-}1 & 0 & \llap{-}1
    \end{bmatrix}
    $
    \item [(D)] $A = 
    \begin{bmatrix}
    1 & 1 & 3 & 2 \\
    0 & 2 & 1 & 1 \\
    \llap{-}1 & 1 & 0 & 1
    \end{bmatrix}
    $
\end{enumerate}
\end{minipage}
\bigskip\\
\textbf{Solution}













\newpage
\problem{13}
Show that if $V$ is a vector space over $F$ and $\vecset{u}{k}$ and $\vecset{v}{l}$ are both basis; then $k = l$.
\bigskip\\
\textbf{Solution}


\par\noindent\rule{\textwidth}{0.4pt}
\bibliographystyle{alpha}
\bibliography{sample}
Slides: Computer Organization and Systems by Prof. Manu Awasthi \\
Book: Computer Organization and Design : the hardware/software interface by David A. Patterson and John L. Hennessy


\end{document}
